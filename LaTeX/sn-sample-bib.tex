\begin{thebibliography}{9}

\bibitem{b1}Tarun, M., Israr, A., \& Gajwal, D. (2022). ANALYSIS OF THE FACTORS INFLUENCE INDIAN ENGLISH ACCENTS, AND HOW PRONUNCIATION AND ARTICULATION FILL THE ACCENT GAP. EPRA International Journal of Environmental Economics, Commerce and Educational Management (ECEM), 9(4), 43–49.
\bibitem{b2} Mridha, M., Ohi, A., Hamid, M., \& Monowar, M. (2022). A study on the challenges and opportunities of speech recognition for Bengali language. Artificial Intelligence Review, 55(4), 3431–3455.
\bibitem{b3} Caballero, M., Moreno, A., \& Nogueiras, A. (2006). Multidialectal acoustic modeling: A comparative study. In Multilingual Speech and Language Processing.
\bibitem{b4} Suman, S., Sahoo, K., Das, C., Jhanjhi, N., \& Mitra, A. (2022). Visualization of Audio Files Using Librosa. In Proceedings of 2nd International Conference on Mathematical Modeling and Computational Science (pp. 409–418).
\bibitem{b5} Bodine, J., \& Hochbaum, D. (2022). A Better Decision Tree: The Max-Cut Decision Tree with Modified PCA Improves Accuracy and Running Time. SN Computer Science, 3(4), 1–18.
\bibitem{b6}Chikkerur, S., \& Ratha, N. (2005). Impact of singular point detection on fingerprint matching performance. In Fourth IEEE Workshop on Automatic Identification Advanced Technologies (AutoID'05) (pp. 207–212).
\bibitem{b7} Upadhyay, R., \& Lui, S. (2018). Foreign English accent classification using deep belief networks. In 2018 IEEE 12th international conference on semantic computing (ICSC) (pp. 290–293).
\bibitem{b8} Ahmed, A., Tangri, P., Panda, A., Ramani, D., \& Karmakar, S. (2019). Vfnet: A convolutional architecture for accent classification. In 2019 IEEE 16th India Council International Conference (INDICON) (pp. 1–4).
\bibitem{b9} Guntur, R., Ramakrishnan, K., \& Mittal, V. (2022). Foreign Accent Recognition Using a Combination of Native and Non-native Speech. In Intelligent Sustainable Systems (pp. 713–721). Springer.
\bibitem{b10} Honnavalli, D., \& Shylaja, S. (2021). Supervised Machine Learning Model for Accent Recognition in English Speech Using Sequential MFCC Features. In Advances in Artificial Intelligence and Data Engineering (pp. 55–66). Springer.
\bibitem{b11} Guntur, R., Ramakrishnan, K., \& Vinay Kumar, M. (2022). An Automated Classification System Based on Regional Accent. Circuits, Systems, and Signal Processing, 41(6), 3487–3507.
\bibitem{b12} Krishna, G., Krishnan, R., \& Mittal, V. (2020). A system for automatic regional accent classification. In 2020 IEEE 17th India Council International Conference (INDICON) (pp. 1–5).
\bibitem{b13} Parikh, P., Velhal, K., Potdar, S., Sikligar, A., \& Karani, R. (2020). English language accent classification and conversion using machine learning. In Proceedings of the International Conference on Innovative Computing & Communications (ICICC).
\bibitem{b14} Purwar, A., Sharma, H., Sharma, Y., Gupta, H., \& Kaur, A. (2022). Accent classification using Machine learning and Deep Learning Models. In 2022 1st International Conference on Informatics (ICI) (pp. 13–18).
\bibitem{b15} Duduka, S., Jain, H., Jain, V., Prabhu, H., \& Chawan, P. (2020). Accent Classification using Machine Learning. International Research Journal of Engineering and Technology (IRJET), 7(11), 638–641.
\bibitem{b16} Pedersen, C., \& Diederich, J. (2007). Accent classification using support vector machines. In 6th IEEE/ACIS International Conference on Computer and Information Science (ICIS 2007) (pp. 444–449).
\bibitem{b17} Badhon, S., Rahaman, H., Rupon, F., \& Abujar, S. (2021). Bengali accent classification from speech using different machine learning and deep learning techniques. In Soft Computing Techniques and Applications (pp. 503–513). Springer.
\bibitem{b18} Mannepalli, K., Sastry, P., \& Suman, M. (2016). MFCC-GMM based accent recognition system for Telugu speech signals. International Journal of Speech Technology, 19(1), 87–93.
\bibitem{b19} Duduka, S., Jain, H., Jain, H., \& Chawan, P. (2021). A Neural Network Approach to Accent Classification. International Research Journal of Engineering and Technology (IRJET), 8(03), 1175–1177.
\bibitem{b20} Hossain, P., Chakrabarty, A., Kim, K., \& Piran, M. (2022). Multi-Label Extreme Learning Machine (MLELMs) for Bangla Regional Speech Recognition. Applied Sciences, 12(11), 5463.
\bibitem{b21} Ahamad, A., Anand, A., \& Bhargava, P. (2020). Accentdb: A database of non-native english accents to assist neural speech recognition. arXiv preprint arXiv:2005.07973.
\bibitem{b22} Althubiti, S., Alenezi, F., Shitharth, S., Reddy, C., \& others (2022). Circuit Manufacturing Defect Detection Using VGG16 Convolutional Neural Networks. Wireless Communications and Mobile Computing, 2022.
\bibitem{b23} Harini, V., \& Bhanumathi, V. (2016). Automatic cataract classification system. In 2016 international conference on communication and signal processing (ICCSP) (pp. 0815–0819).
\end{thebibliography}

